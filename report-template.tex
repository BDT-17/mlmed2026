\documentclass[conference]{IEEEtran}

\title{ECG Heartbeat Classification using PyTorch}

\author{
    Bui Duc Thang -- 23BI14400 \\
    Department of ICT \\
   
}

\begin{document}

\maketitle

\begin{abstract}
This report presents the implementation of a deep learning model for ECG heartbeat 
classification using the MIT-BIH Arrhythmia dataset. A feed-forward neural network 
was trained on ECG beat segments to classify heartbeats into five categories. 
Performance was evaluated using accuracy and macro F1-score, demonstrating strong 
generalization and steady improvement across epochs.
\end{abstract}

\section{Introduction}
Electrocardiogram (ECG) signals are widely used for diagnosing cardiac conditions. 
Automated classification of ECG beats can assist clinicians by reducing manual workload 
and improving diagnostic accuracy. In this project, we implement a PyTorch-based 
feed-forward neural network to classify ECG beats into five categories.

\section{Dataset}
The dataset used is the MIT-BIH Arrhythmia dataset, provided in CSV format:
\begin{itemize}
    \item \textbf{Training set:} \texttt{mitbih\_train.csv}
    \item \textbf{Test set:} \texttt{mitbih\_test.csv}
\end{itemize}

Each row corresponds to one ECG beat segment of length 187 samples, followed by a label 
indicating the heartbeat class (0–4). The dataset contains five heartbeat categories:
Normal, Supraventricular, Ventricular, Fusion, and Unknown.

\section{Model Implementation}
We implemented a feed-forward neural network in PyTorch:
\begin{itemize}
    \item Input dimension: 187
    \item Hidden layers: Two fully connected layers with GELU activation
    \item Output: 5-class logits
\end{itemize}

The model is trained using the Adam optimizer and cross-entropy loss.

\section{Training Procedure}
\begin{itemize}
    \item Batch size: 64
    \item Epochs: 10
    \item Learning rate: $1 \times 10^{-4}$
    \item Device: CPU
\end{itemize}

The training loop computes average training loss per epoch. Validation is performed 
after each epoch, with accuracy and macro F1-score reported.

\section{Results}
Table~\ref{tab:training} shows the training progress over 10 epochs, including 
training loss, validation accuracy, and validation F1-score.

\begin{table}[h]
\centering
\caption{Training Progress over 10 Epochs}
\label{tab:training}
\begin{tabular}{|c|c|c|c|}
\hline
\textbf{Epoch} & \textbf{Train Loss} & \textbf{Val Acc} & \textbf{Val F1} \\
\hline
01 & 0.4730 & 0.9081 & 0.4611 \\
02 & 0.2896 & 0.9294 & 0.5554 \\
03 & 0.2361 & 0.9430 & 0.7021 \\
04 & 0.2079 & 0.9453 & 0.7132 \\
05 & 0.1900 & 0.9488 & 0.7255 \\
06 & 0.1775 & 0.9524 & 0.7519 \\
07 & 0.1680 & 0.9551 & 0.7781 \\
08 & 0.1623 & 0.9555 & 0.7649 \\
09 & 0.1557 & 0.9578 & 0.7865 \\
10 & 0.1499 & 0.9592 & 0.7945 \\
\hline
\end{tabular}
\end{table}

The model demonstrates steady improvement, with validation accuracy reaching 
95.9\% and macro F1-score approaching 0.79 by epoch 10.

\section{Repository Workflow}
The following files were pushed to the forked GitHub repository:
\begin{itemize}
    \item \texttt{Report.1.tex} (this report)
    
    \item \texttt{practice1.py} 
\end{itemize}



\section{Conclusion}
This project demonstrates ECG heartbeat classification using a simple feed-forward 
network. Future work includes experimenting with convolutional architectures, 
transfer learning, and per-class evaluation metrics to further improve performance.

\bibliographystyle{IEEEtran}
\bibliography{references}
https://www.kaggle.com/datasets/shayanfazeli/heartbeat/data
\end{document}
